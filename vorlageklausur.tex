% vim: set spelllang=de,en:
\documentclass[a4paper, 11pt, oneside]{scrartcl}
\usepackage[utf8]{inputenc}

% -----------------------------------------------------------------------------
% Hier bitte Daten eintragen
% Zeilen von nicht passenden oder nicht vorhandenen Daten bitte löschen oder auskommentieren
% -----------------------------------------------------------------------------
\newcommand{\Titel}        {<Vorlesungsname>} % hier sinnvollen Namen wie Geometrie, Lineare Algebra, ...
\newcommand{\Modul}        {<MODUL>} % hier die Modulbeschreibungsabkürzung wie GEO, ANAG, LAAG, HANA, ...
\newcommand{\Prueferin}      {<Prüferin>}
\newcommand{\Beisitzerin}    {<Beisitzerin>} % wenn unbekannt und bei schriftlichen Klausuren löschen
\newcommand{\Kursassistenz}{<Kursassistentin>} % wenn unbekannt, zei mündlichen Prüfungen löschen
\newcommand{\Semester}     {<WS 2016/ 17>} % oder SS 2017, ... oder WS/ SS 2016/ 17; das Semester in der die Vorlesung gehalten wurde
\newcommand{\Zeit}         {<90 min>} % zeitliche Länge der Prüfung
\newcommand{\Studiengang}  {<Kurzform>} % Ba Mathe, Ma Mathe, LA Gy, LA MS, LA BBS, LA GS, Dipl. Mathe
\newcommand{\Hilfsmittel}  {<1 Seite handbeschrieben>} % was du in die Prüfung mitnehmen durftest
% \newcommand{\weiteres}     {Infobezeichnung 1 & Wert 1\\
                              % Infobezeichnung 2 & Wert 2} % wenn es weitere interessante Infos gibt, die oben stehen sollen
  % hier auskommentiert, da die meisten Prüfungen mit den Daten oben auskommen sollten
% 

% -----------------------------------------------------------------------------
% Diesen Teil musst du nicht ansehen. Er funktioniert halt.
% Falls das Dokument bei dir nicht kompiliert, weil der Header Probleme macht,
% ärgere dich nicht damit rum.
% Sende uns einfach die .tex - Datei und wir fixen das.
% -----------------------------------------------------------------------------

\usepackage[ngerman]{babel}
\usepackage{csquotes, ngerman, scrpage2, amsmath, amsthm,amssymb} %  calc,
% \usepackage[pdftex, debugshow, final]{graphicx}
\pagestyle{scrheadings}

\begin{document}

\footskip=1.5cm
\oddsidemargin=.5cm
\voffset=-1cm
\topmargin=1cm
\headsep=1cm 
\linespread{1.3} 
 
\chead{\sffamily Fachschaftsrat Mathematik - Prüfungsprotokoll}
\setheadsepline{.5pt}
% \ofoot{\sffamily\thepage}
\setfootsepline{.5pt}

\begin{tabular}{|p{0.45\textwidth}|p{0.45\textwidth}|}
\hline
  \begin{minipage}[c]{0.45\textwidth} %[4.2cm]
  \vskip1em
  \begin{center}
    \textbf{\Large \ifdefined\Titel\Titel\else Vorlesungstitel eintragen!\fi} \\ \hrulefill % 
  \end{center}
  \begin{tabular*}{\linewidth}{l@{\extracolsep\fill}r}
    \ifdefined\Modul Modul:&\Modul\\\fi
    \ifdefined\Prueferin Prüfer:in:&\Prueferin\\\fi
    \ifdefined\Beisitzerin Beisitzerin:&\Beisitzerin\\\fi
    \ifdefined\Kursassistenz Kursassistenz:&\Kursassistenz\\\fi
    \ifdefined\Semester Semester:&\Semester\\\fi
    \ifdefined\Zeit Zeit:&\Zeit\\\fi
    \ifdefined\Studiengang Studiengang:&\Studiengang\\\fi
    \ifdefined\Note Note:&\Note\\\fi
    \ifdefined\weiteres \weiteres\fi
    \end{tabular*}
  \end{minipage}
  &
   \begin{minipage}[c]{0.45\textwidth} %[4.2cm] [\ht\infobox+\dp\infobox]
   \vskip0.5em
    \linespread{1} \scriptsize Liebe Leserin dieses Protokolls!\\
    Die Protokolle des $\mu$FSR dürfen nur unverändert weitergegeben werden. Falls du Fehler finden solltest oder Verbesserungsvorschläge hast, dann schicke bitte eine E-Mail an {\sffamily klausur@myfsr.de}. Bitte unterstütze diesen Service des $\mu$FSR und schreibe ebenfalls ein Prüfungsprotokoll nach deiner Prüfung. Du kannst dabei entweder eine Textdatei oder (wenn du uns Arbeit abnehmen möchtest) eine \LaTeX-Datei schicken. Die Vorlage findest du auf {\sffamily github.com/myfsr/TeXVorlagen} %Den Link zur Vorlage findest du auf {\sffamily myfsr.de}
    \\
    \rightline{Vielen Dank: Euer $\mu$FSR}\\
  \end{minipage}\\
  \hline
\end{tabular}
\vskip0.5em

\ifdefined\Hilfsmittel Erlaubte Hilfsmittel: \Hilfsmittel.\fi

\newcommand{\points}[1]{[\textbf{#1}]}
% -----------------------------------------------------------------------------
% Ende des Headers
% Ab hier schreibst du deinen Text.
% Du kannst den untenstehenden Vorschlag zur Gliederung übernehmen, wenn er in etwa der Prüfung entspricht.
% -----------------------------------------------------------------------------

\begin{enumerate}
  \item Diesen Text durch den Aufgabentext der ersten Aufgabe ersetzen. Nummerierung passiert automatisch.
  \begin{enumerate}
    \item \points{3} Falls Aufgabe in a), b), ... unteteilt, hier Aufgabenteil a) der ersten Aufgabe. Nummerierung mit a), b), ... passiert automatisch. Dieser Aufgabenteil bringt 3 Punkte.
    \begin{align*}
      \int_a^b \frac {\sin x}x dx = \text{?}
    \end{align*}
  \end{enumerate}
\end{enumerate}

\section*{Ein Thema}
Bei kombinierten Prüfungen ist es meist sinnvoll diese mit sections zu trennen.
\begin{itemize}
  \item Der eine wichtige Satz.
  \item Der andere wichtige Satz.
\end{itemize}

\section*{Eindruck}
Gerade bei mündlichen Prüfungen ist es meist am interessantesten zu lesen, wie die Prüferin so ist, welche Atmosphäre in der Prüfung herrscht und was ihr wichtig ist.
\end{document}
